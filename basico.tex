\section{Configurando o Git}

\subsection {Primeira vez}

Configurar o nome do usu�rio:

No terminal:

\textit{\$ git config --global user.name `Seu nome` }

\textit{\$ git config --global user.email voce@seudominio.exemplo.com }

Colocar cores:

\textit{\$ git config --global color.diff auto }

\textit{\$ git config --global color.status auto }

\textit{\$ git config --global color.branch auto }

Habilitar git-rerere (opcional):

\textit{\$ git config --global rerere.enable 1 }

\subsection{Como baixar um reposit�rio remoto}

\textit{\$ cd ~/wa-git}  

Em que wa-git representa o diret�rio de trabalho.

\textit{\$ git clone <URL do reposit�rio>}

Exemplo:

<<<<<<< HEAD
\textit{\$ git clone ssh://michelfarah@zootecnista.com.br/<diret�rio do
reposit�rio>}

=======
\textit{\$ git clone ssh://michelfarah@zootecnista.com.br}
>>>>>>> 753fbd3cf3f3bbe988b879abfd78ff1336d22368
